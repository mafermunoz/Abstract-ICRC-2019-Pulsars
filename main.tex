\documentclass{article}
\usepackage[utf8]{inputenc}

\title{Abstract ICRC 2019: $\gamma$-ray Pulsars with DAMPE}
\author{Maria Muñoz Salinas}
\date{December 2018}

\begin{document}

\maketitle

\begin{abstract}

    
    
    The DArk Matter Particle Explorer (DAMPE) is a satellite-borne experiment successfully launched in December 2015.  DAMPE has been taking data for over 3 years during which it has been observing the full gamma-ray sky above 1 GeV. The data used for this contribution considers the first 2 years of data and within an energy range from 2 GeV to 100 GeV.   Among all the pulsars observed by DAMPE we will present the measurement results for the 10 brightest gamma-ray pulsars which include Vela, Geminga and the Crab. The light curves and energy spectra, including the phase averaged energy spectra will be presented for these 10 pulsars. From these results we will derive the timing precision of DAMPE, which is of importance for its other scientific goals, as well as the potential for future gamma-ray studies with DAMPE.
\end{abstract}
\end{document}
